% ----------------------- TODO ---------------------------
% Diese Daten müssen pro Blatt angepasst werden:
\newcommand{\NUMBER}{6}
\newcommand{\EXERCISES}{2}
% Diese Daten müssen einmalig pro Vorlesung angepasst werden:
\newcommand{\COURSE}{Grundlagen des Internets}
\newcommand{\TUTOR}{}
\newcommand{\STUDENTA}{Moritz Hahn}
\newcommand{\STUDENTB}{Sarah Altenkrüger}
\newcommand{\STUDENTC}{}
\newcommand{\DEADLINE}{03.06.2024}
% ----------------------- TODO ---------------------------

\documentclass[a4paper]{scrartcl}

\usepackage[utf8]{inputenc}
\usepackage[ngerman]{babel}
\usepackage{amsmath}
\usepackage{amssymb}
\usepackage{fancyhdr}
\usepackage{color}
\usepackage{graphicx}
\usepackage{lastpage}
\usepackage{listings}
\usepackage{tikz}
\usepackage{pdflscape}
\usepackage{subfigure}
\usepackage{float}
\usepackage{polynom}
\usepackage{hyperref}
\usepackage{tabularx}
\usepackage{forloop}
\usepackage{geometry}
\usepackage{listings}
\usepackage{fancybox}
\usepackage{tikz}
\usepackage{comment}
%Größe der Ränder setzen
\geometry{a4paper,left=3cm, right=3cm, top=3cm, bottom=3cm}

%Kopf- und Fußzeile
\pagestyle {fancy}
\fancyhead[L]{Tutor: \TUTOR}
\fancyhead[C]{\COURSE}
\fancyhead[R]{\today}

\fancyfoot[L]{}
\fancyfoot[C]{} 
\fancyfoot[R]{Seite \thepage /\pageref*{LastPage}}

%Formatierung der Überschrift, hier nichts ändern
\def\header#1#2{
  \begin{center}
    {\Large Übungsblatt #1}\\
    {(Abgabetermin #2)}
  \end{center}
}

%Definition der Punktetabelle, hier nichts ändern
\newcounter{punktelistectr}
\newcounter{punkte}
\newcommand{\punkteliste}[2]{%
  \setcounter{punkte}{#2}%
  \addtocounter{punkte}{-#1}%
  \stepcounter{punkte}%<-- also punkte = m-n+1 = Anzahl Spalten[1]
  \begin{center}%
  \begin{tabularx}{\linewidth}[]{@{}*{\thepunkte}{>{\centering\arraybackslash} X|}@{}>{\centering\arraybackslash}X}
      \forloop{punktelistectr}{#1}{\value{punktelistectr} < #2 } %
      {%
        \thepunktelistectr &
      }
      #2 &  $\Sigma$ \\
      \hline
      \forloop{punktelistectr}{#1}{\value{punktelistectr} < #2 } %
      {%
        &
      } &\\
      \forloop{punktelistectr}{#1}{\value{punktelistectr} < #2 } %
      {%
        &
      } &\\
    \end{tabularx}
  \end{center}
}

\begin{document}

\begin{tabularx}{\linewidth}{m{0.2 \linewidth}X}
  \begin{minipage}{\linewidth}
    \STUDENTA\\
    \STUDENTB\\
    \STUDENTC
  \end{minipage} & \begin{minipage}{\linewidth}
    \punkteliste{1}{\EXERCISES}
  \end{minipage}\\
\end{tabularx}

\begin{comment}
Autonomes system 1 (AS1):
1a-1c
1a-1d über IF1
1a-1b
1d-1b über IF2
1c-3a
1b-2a

Autonomes System 2 (AS2):
2a-1b
2a-2c
2a-2b
2b-2c
2c-4a

Autonomes System 3 (AS3):
3a-1c
3a-3b
3a-3c
3b-3c
3c-4c

Autonomes System 4 (AS4):
4c-3c
4c-4b
4b-4a
4a-2c
4a-X

Autonomes system 1 (AS1):
R1-R2
R1-134.0.0.0/8

Autonomes System 2 (AS2):
R2-R1 
R2-R5
R2-R3

Autonomes System 3 (AS3):
R3-R2
R3-R5
R3-R4

Autonomes System 4 (AS4):
R4-R3
R4-R5

Autonomes System 5 (AS5):
R5-R4
R5-R3
R5-R2
r5-R6

Autonomes System 6 (AS6):
R6-R5
\end{comment}
\header{Nr. \NUMBER}{\DEADLINE}



% ----------------------- TODO ---------------------------
% Hier werden die Aufgaben/Lösungen eingetragen:

\section*{Aufgabe 6.1}
\subsection*{1.}
Zunächst betrachten wir, wie sich die Information über das Präfix \( x \) verbreitet:

\begin{itemize}
    \item \textbf{Router 4a} in AS4 kennt das Präfix \( x \) und verteilt es über \textbf{RIP} innerhalb von AS4 zu \textbf{4b} und \textbf{4c}.
    \item \textbf{Router 4c} in AS4 teilt die Information über \( x \) mit \textbf{Router 3c} in AS3 über \textbf{eBGP}.
    \item \textbf{Router 3c} teilt die Information über \( x \) in AS3 über \textbf{iBGP} mit \textbf{3b} und \textbf{3a}.
    \item \textbf{Router 3a} teilt die Information über \( x \) mit \textbf{1c} in AS1 über \textbf{eBGP}.
    \item \textbf{Router 1c} teilt die Information über \( x \) in AS1 über \textbf{iBGP} mit \textbf{1d}.
\end{itemize}

Damit können wir die Frage beantworten:

\begin{itemize}
    \item \textbf{Router 1c}: erfährt die Information über \( x \) über \textbf{eBGP} (von \textbf{3a}).
    \item \textbf{Router 3c}: erfährt die Information über \( x \) über \textbf{eBGP} (von \textbf{4c}).
    \item \textbf{Router 1d}: erfährt die Information über \( x \) über \textbf{iBGP} (von \textbf{1c}).
    \item \textbf{Router 3a}: erfährt die Information über \( x \) über \textbf{iBGP} (von \textbf{3c}).
\end{itemize}
\subsection*{2.}
Um zu bestimmen, ob \( I \) auf IF1 oder IF2 gesetzt wird, müssen wir den Weg des Pakets von \textbf{1d} zum Präfix \( x \) verfolgen:

\begin{itemize}
    \item \textbf{Router 1d} erhält die Route zu \( x \) von \textbf{1c}.
    \item \textbf{Router 1c} hat die Route zu \( x \) von \textbf{3a}.
    \item Die Kommunikation zwischen \textbf{1c} und \textbf{3a} erfolgt über ihre direkte Verbindung.
\end{itemize}

Da der Weg vom Router \textbf{1d} zum Präfix \( x \) über den Router \textbf{1c} führt und die Verbindung von \textbf{1d} zu \textbf{1c} über das Interface IF1 läuft, wird \( I \) auf \textbf{IF1} gesetzt.

\textbf{Begründung}: Da der nächste Hop für Router 1d, um das Präfix \( x \) zu erreichen, über Router 1c erfolgt und die Verbindung von 1d zu 1c über IF1 läuft, wird \( I \) auf IF1 gesetzt.

\subsection*{3.}
Mit der neuen Verbindung zwischen AS2 und AS4 (2c-4a) erhält \textbf{Router 1d} zwei verschiedene Routen zu \( x \): eine über AS3 und eine über AS2. 
Welche Route \textbf{Router 1d} bevorzugt, hängt von den BGP-Pfadattributen ab, wie z.B. der Länge des AS-Pfads.

\begin{itemize}
    \item Der Pfad über AS3: 1d → 1c → 3a → 3c → 4c → 4a → \( x \)
    \item Der Pfad über AS2: 1d → 1b → 2a → 2c → 4a → \( x \)
\end{itemize}

Da BGP standardmäßig den Pfad mit der geringsten AS-Pfadlänge bevorzugt und der Pfad über AS2 kürzer ist (2 AS-Hops gegenüber 3 AS-Hops über AS3), wird \textbf{Router 1d} den Pfad über AS2 wählen.

Damit wird \( I \) auf \textbf{IF2} gesetzt, weil der Router 1d über 1b (und damit IF2) auf \( x \) zugreifen wird.

\textbf{Begründung}: Da der Pfad über AS2 kürzer ist (weniger AS-Hops), wird \textbf{Router 1d} den Pfad über AS2 bevorzugen, und \( I \) wird auf IF2 gesetzt.

\subsection*{4.}
\begin{itemize}
  \item Der Pfad über AS3: 1d → 1c → 3a → 3c → 4c → 4a → \( x \) (3 AS-Hops)
  \item Der Pfad über AS2: 1d → 1b → 2a → 2c → 5a → 4a → \( x \) (4 AS-Hops)
\end{itemize}

Da BGP standardmäßig den Pfad mit der geringsten AS-Pfadlänge bevorzugt und der Pfad über AS3 kürzer ist (3 AS-Hops gegenüber 4 AS-Hops über AS2:AS5:AS4), wird \textbf{Router 1d} den Pfad über AS3 wählen.

Damit wird \( I \) auf \textbf{IF1} gesetzt, weil der Router 1d über 1c (und damit IF1) auf \( x \) zugreifen wird.

\textbf{Begründung}: Da der Pfad über AS3 kürzer ist (weniger AS-Hops), wird \textbf{Router 1d} den Pfad über AS3 bevorzugen, und \( I \) wird auf IF1 gesetzt.

\section*{Aufgabe 6.2}
\subsection*{1.}
\subsection*{Runde 1:}

\begin{itemize}
    \item \textbf{R1 → R2:}
    \begin{itemize}
        \item \textbf{Announce:}
        \begin{itemize}
            \item Präfix: 134.0.0.0/8
            \item NextHop: R1
            \item ASPath: [AS1]
        \end{itemize}
    \end{itemize}
\end{itemize}

\subsection*{Runde 2:}

\begin{itemize}
    \item \textbf{R2 → R5:}
    \begin{itemize}
        \item \textbf{Announce:}
        \begin{itemize}
            \item Präfix: 134.0.0.0/8
            \item NextHop: R2
            \item ASPath: [AS1, AS2]
        \end{itemize}
    \end{itemize}
    \item \textbf{R2 → R3:}
    \begin{itemize}
        \item \textbf{Announce:}
        \begin{itemize}
            \item Präfix: 134.0.0.0/8
            \item NextHop: R2
            \item ASPath: [AS1, AS2]
        \end{itemize}
    \end{itemize}
\end{itemize}

\subsection*{Runde 3:}

\begin{itemize}
    \item \textbf{R5 → R3:}
    \begin{itemize}
        \item \textbf{Announce:}
        \begin{itemize}
            \item Präfix: 134.0.0.0/8
            \item NextHop: R5
            \item ASPath: [AS1, AS2, AS5]
        \end{itemize}
    \end{itemize}
    \item \textbf{R5 → R4:}
    \begin{itemize}
        \item \textbf{Announce:}
        \begin{itemize}
            \item Präfix: 134.0.0.0/8
            \item NextHop: R5
            \item ASPath: [AS1, AS2, AS5]
        \end{itemize}
    \end{itemize}
    \item \textbf{R5 → R6:}
    \begin{itemize}
        \item \textbf{Announce:}
        \begin{itemize}
            \item Präfix: 134.0.0.0/8
            \item NextHop: R5
            \item ASPath: [AS1, AS2, AS5]
        \end{itemize}
    \end{itemize}
    \item \textbf{R3 → R4:}
    \begin{itemize}
        \item \textbf{Announce:}
        \begin{itemize}
            \item Präfix: 134.0.0.0/8
            \item NextHop: R3
            \item ASPath: [AS1, AS2, AS3]
        \end{itemize}
    \end{itemize}
\end{itemize}

\subsection*{Runde 4:}

\begin{itemize}
    \item \textbf{R4 → R5:}
    \begin{itemize}
        \item \textbf{Announce:}
        \begin{itemize}
            \item Präfix: 134.0.0.0/8
            \item NextHop: R4
            \item ASPath: [AS1, AS2, AS3, AS4]
        \end{itemize}
    \end{itemize}
\end{itemize}

\subsection*{Zusammenfassung der Nachrichten:}

\begin{itemize}
    \item \textbf{Announce:}
    \begin{itemize}
        \item \textbf{Runde 1:} 1 Nachricht (R1 → R2)
        \item \textbf{Runde 2:} 2 Nachrichten (R2 → R5, R2 → R3)
        \item \textbf{Runde 3:} 4 Nachrichten (R5 → R3, R5 → R4, R5 → R6, R3 → R4)
        \item \textbf{Runde 4:} 1 Nachricht (R4 → R5)
    \end{itemize}
\end{itemize}

Insgesamt: 8 Announce-Nachrichten.

\subsection*{2.}
\subsection*{R5 kennt folgende Routen zum Präfix 134.0.0.0/8:}

\begin{enumerate}
    \item 134.0.0.0/8 über R2, ASPath: [AS1, AS2]
    \item 134.0.0.0/8 über R3, ASPath: [AS1, AS2, AS3]
    \item 134.0.0.0/8 über R4, ASPath: [AS1, AS2, AS3, AS4]
\end{enumerate}

\subsection*{R5 gibt folgende Route gegenüber R6 bekannt:}

\begin{itemize}
    \item \textbf{Präfix:} 134.0.0.0/8
    \item \textbf{NextHop:} R5
    \item \textbf{ASPath:} [AS1, AS2, AS5]
\end{itemize}

\textbf{Begründung:} R5 wählt die Route mit dem kürzesten ASPath, und bei gleich langen ASPaths die Route mit der kleinsten Router-ID (R2).

\subsection*{3.}
\begin{itemize}
  \item \textbf{R2 sendet Withdraw-Nachrichten an R3:}
  \begin{itemize}
      \item \textbf{Withdraw:}
      \begin{itemize}
          \item Präfix: 134.0.0.0/8
      \end{itemize}
  \end{itemize}
  \item \textbf{R5 sendet Withdraw-Nachrichten an R3, R4 und R6:}
  \begin{itemize}
      \item \textbf{Withdraw:}
      \begin{itemize}
          \item Präfix: 134.0.0.0/8
      \end{itemize}
  \end{itemize}
\end{itemize}

\subsection*{3.}
\begin{itemize}
  \item \textbf{R2 sendet Withdraw-Nachrichten an R5:}
  \begin{itemize}
      \item \textbf{Withdraw:}
      \begin{itemize}
          \item Präfix: 134.0.0.0/8
      \end{itemize}
  \end{itemize}
  \item \textbf{R3 sendet Withdraw-Nachrichten an R5 und R4:}
  \begin{itemize}
      \item \textbf{Withdraw:}
      \begin{itemize}
          \item Präfix: 134.0.0.0/8
      \end{itemize}
  \end{itemize}
  \item \textbf{R5 sendet Withdraw-Nachrichten an R6:}
  \begin{itemize}
      \item \textbf{Withdraw:}
      \begin{itemize}
          \item Präfix: 134.0.0.0/8
      \end{itemize}
  \end{itemize}
\end{itemize}
\end{document}
%%% Local Variables:
%%% mode: latex
%%% TeX-master: t
%%% End:
