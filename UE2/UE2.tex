% ----------------------- TODO ---------------------------
% Diese Daten müssen pro Blatt angepasst werden:
\newcommand{\NUMBER}{2}
\newcommand{\EXERCISES}{5}
% Diese Daten müssen einmalig pro Vorlesung angepasst werden:
\newcommand{\COURSE}{Grundlagen des Internets}
\newcommand{\TUTOR}{}
\newcommand{\STUDENTA}{Moritz Hahn}
\newcommand{\STUDENTB}{Sarah Altenkrüger}
\newcommand{\STUDENTC}{}
\newcommand{\DEADLINE}{06.05.2024}
% ----------------------- TODO ---------------------------

\documentclass[a4paper]{scrartcl}

\usepackage[utf8]{inputenc}
\usepackage[ngerman]{babel}
\usepackage{amsmath}
\usepackage{amssymb}
\usepackage{fancyhdr}
\usepackage{color}
\usepackage{graphicx}
\usepackage{lastpage}
\usepackage{listings}
\usepackage{tikz}
\usepackage{pdflscape}
\usepackage{subfigure}
\usepackage{float}
\usepackage{polynom}
\usepackage{hyperref}
\usepackage{tabularx}
\usepackage{forloop}
\usepackage{geometry}
\usepackage{listings}
\usepackage{fancybox}
\usepackage{tikz}
\usepackage{tabto}
%Größe der Ränder setzen
\geometry{a4paper,left=3cm, right=3cm, top=3cm, bottom=3cm}

%Kopf- und Fußzeile
\pagestyle {fancy}
\fancyhead[L]{Tutor: \TUTOR}
\fancyhead[C]{\COURSE}
\fancyhead[R]{\today}

\fancyfoot[L]{}
\fancyfoot[C]{} 
\fancyfoot[R]{Seite \thepage /\pageref*{LastPage}}

%Formatierung der Überschrift, hier nichts ändern
\def\header#1#2{
  \begin{center}
    {\Large Übungsblatt #1}\\
    {(Abgabetermin #2)}
  \end{center}
}

%Definition der Punktetabelle, hier nichts ändern
\newcounter{punktelistectr}
\newcounter{punkte}
\newcommand{\punkteliste}[2]{%
  \setcounter{punkte}{#2}%
  \addtocounter{punkte}{-#1}%
  \stepcounter{punkte}%<-- also punkte = m-n+1 = Anzahl Spalten[1]
  \begin{center}%
  \begin{tabularx}{\linewidth}[]{@{}*{\thepunkte}{>{\centering\arraybackslash} X|}@{}>{\centering\arraybackslash}X}
      \forloop{punktelistectr}{#1}{\value{punktelistectr} < #2 } %
      {%
        \thepunktelistectr &
      }
      #2 &  $\Sigma$ \\
      \hline
      \forloop{punktelistectr}{#1}{\value{punktelistectr} < #2 } %
      {%
        &
      } &\\
      \forloop{punktelistectr}{#1}{\value{punktelistectr} < #2 } %
      {%
        &
      } &\\
    \end{tabularx}
  \end{center}
}

\begin{document}

\begin{tabularx}{\linewidth}{m{0.2 \linewidth}X}
  \begin{minipage}{\linewidth}
    \STUDENTA\\
    \STUDENTB\\
    \STUDENTC
  \end{minipage} & \begin{minipage}{\linewidth}
    \punkteliste{1}{\EXERCISES}
  \end{minipage}\\
\end{tabularx}

\header{Nr. \NUMBER}{\DEADLINE}



% ----------------------- TODO ---------------------------
% Hier werden die Aufgaben/Lösungen eingetragen:

\section*{Aufgabe 2.1}
\begin{table}[!htb]
\centering
\caption{R1}
\begin{tabular}{|l|l|l|l|}
\hline
In-Port & In-Label & Out-Port & Out-Label \\ \hline
-       & -        & IF0      & L1        \\ \hline
IF0     & L5       & -        & -         \\ \hline
\end{tabular}
\end{table}

\begin{table}[!htb]
\centering
\caption{R2}
\begin{tabular}{|l|l|l|l|}
\hline
In-Port & In-Label & Out-Port & Out-Label \\ \hline
-       & -        & IF0      & L4        \\ \hline
IF0     & L6       & -        & -         \\ \hline
\end{tabular}
\end{table}

\begin{table}[!htb]
\centering
\caption{R3}
\begin{tabular}{|l|l|l|l|}
\hline
In-Port & In-Label & Out-Port & Out-Label \\ \hline
IF0     & L2       & -        & -         \\ \hline
\end{tabular}
\end{table}

\begin{table}[!htb]
\centering
\caption{R4}
\begin{tabular}{|l|l|l|l|}
\hline
In-Port & In-Label & Out-Port & Out-Label \\ \hline
IF0     & L1       & IF2      & L3        \\ \hline
IF2     & L1       & IF0      & L5        \\ \hline
IF2     & L1       & IF1      & L6        \\ \hline
IF1     & L4       & IF2      & L3        \\ \hline
\end{tabular}
\end{table}

\begin{table}[!htb]
\centering
\caption{R5}
\begin{tabular}{|l|l|l|l|}
\hline
In-Port & In-Label & Out-Port & Out-Label \\ \hline
IF0     & L3       & IF2      & L2        \\ \hline
IF2     & L3       & IF0      & L1        \\ \hline
IF2     & L1       & IF1      & L2        \\ \hline
\end{tabular}
\end{table}


\begin{table}[h]
\centering
\caption{R6}
\begin{tabular}{|l|l|l|l|}
\hline
In-Port & In-Label & Out-Port & Out-Label \\ \hline
IF0     & L2       & -        & -         \\ \hline
-       & -        & IF0      & L3        \\ \hline
-       & -        & IF0      & L1        \\ \hline
\end{tabular}
\end{table}
\newpage
\section*{Aufgabe 2.2}
\subsection*{1.)}
A: L0 \\
B: L1 \\
C: L2 \\
D: L3 
\subsection*{2.)}
A: L0 \\
B: L1 \\
C: L2 \\
D: L3 
\subsection*{3.)}
A: L0 \\
B: L1, L0 \\
C: L2, L 0 \\
D: L3 
\subsection*{4.)}
A: L0 \\
B: L1, L0 \\
C: L2, L0 \\
D: L3 

\section*{Aufgabe 2.3}
\subsection*{1.)}
Routing bezieht sich auf den Prozess der Bestimmung des besten Pfades für den Datenverkehr von einer Quelle zu einem Ziel über ein Netzwerk. 
Dies geschieht anhand von Routing-Algorithmen und Routing-Tabellen, die Informationen über die verfügbaren Wege und deren Metriken enthalten. \\
Forwarding hingegen ist der Prozess, bei dem tatsächlich Datenpakete gemäß den Einträgen in der Routing-Tabelle weitergeleitet werden. 
Es ist die physische Aktion, die Daten von einem Interface zum anderen bewegt, basierend auf den Informationen aus der Routing-Tabelle. 
Forwarding ist eher eine Funktion der Hardware (z. B. Router), während Routing eher eine Funktion der Software ist.
\subsection*{2.)}
Netze werden in Subnetze unterteilt, um die Effizienz der Adressnutzung zu erhöhen und die Verwaltung von IP-Adressen zu vereinfachen. \\
\underline{Vorteil:} Subnetze ermöglichen eine bessere Nutzung von IP-Adressen, da sie es ermöglichen, große Adressbereiche in kleinere, handhabbare Gruppen aufzuteilen. \\
Dies reduziert die Anzahl der benötigten Broadcast-Domänen und minimiert die Größe von Routing-Tabellen, was die Netzwerkleistung verbessert.\\
\underline{Nachteil:} Die Unterteilung von Netzen in Subnetze kann zu einem erhöhten Verwaltungsaufwand führen, insbesondere wenn viele Subnetze eingerichtet werden müssen. \\
Es erfordert sorgfältige Planung und Konfiguration, um sicherzustellen, dass die Subnetze angemessen dimensioniert sind und dass die Routing- und Subnetzmasken korrekt konfiguriert sind.
\subsection*{2.)}
1.0.0.0/31

\section*{Aufgabe 2.4}
\subsection*{1.)}
Um die Netzwerk-ID des Subnetzes zu bestimmen, müssen wir die Subnetzmaske aus der CIDR-Notation extrahieren und dann die Bitweise UND-Operation auf die IP-Adresse und die Subnetzmaske anwenden. \\
Die gegebene IP-Adresse: 10.0.13.42/17\\
Die Subnetzmaske für /17 in binärer Form: 11111111.11111111.10000000.00000000 (oder 255.255.128.0 in dezimaler Form)\\
IP-Adresse:  \tab   00001010.00000000.00001101.00101010 \\
Subnetzmaske: \tab   11111111.11111111.10000000.00000000 \\
-------------------------------------------------   \\
Netzwerk-ID:   \tab  00001010.00000000.00000000.00000000

\subsection*{2.)}
Um die Anzahl der nutzbaren IP-Adressen in diesem Subnetz zu bestimmen, müssen wir die Anzahl der Hostbits berechnen. Da die Subnetzmaske /17 ist, bleiben 15 Bits für Hostadressen übrig (32 - 17 = 15).\\
Die Anzahl der nutzbaren IP-Adressen berechnet sich durch $2^{15} -2$, da die erste Adresse für das Netzwerk und die letzte Adresse für den Broadcast reserviert sind.\\
Anzahl der nutzbaren IP-Adressen = $2^{15} - 2 = 32766$ \\
Die erste nutzbare IP-Adresse ist die Netzwerk-ID plus 1 und die letzte nutzbare IP-Adresse ist die Broadcast-Adresse minus 1.\\
Erste nutzbare IP-Adresse: 10.0.0.1\\
Letzte nutzbare IP-Adresse: 10.0.127.254
\subsection*{3.)}
Ja, Hosts mit den IP-Adressen 10.0.48.32/17 und 10.0.13.42/17 können kommunizieren, da sie sich im gleichen Subnetz befinden. Die Subnetzmaske (/17) definiert das Subnetz, und beide Hosts haben die gleiche Netzwerk-ID (10.0.0.0), was bedeutet, dass sie sich im selben Subnetz befinden und daher direkt miteinander kommunizieren können.
\subsection*{4.)}
Nein, Hosts mit den IP-Adressen 10.67.87.54/17 und 10.0.13.42/17 können nicht kommunizieren, da sie sich in unterschiedlichen Subnetzen befinden. Um die Kommunikation zu ermöglichen, müsste eine Router-Komponente im Netzwerk vorhanden sein, die Routingfunktionen zwischen den Subnetzen durchführt.

\section*{Aufgabe 2.4}
\subsection*{1.)}
5.45.105.247
\subsection*{2.)}
10.0.0.0/8, da ansonsten nur weniger als 4 subnetze erzeugt werden können da die Subnetzmaske 255.192.0.0 ist
\subsection*{3.)}
\end{document}
%%% Local Variables:
%%% mode: latex
%%% TeX-master: t
%%% End:
